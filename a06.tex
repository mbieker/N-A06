\documentclass[11pt,ngerman,a4paper]{article}
%Gummi|061|=)
\usepackage{amsmath}
\usepackage{a4wide}

\usepackage{amsthm}
\usepackage{amsbsy}
\usepackage{amssymb}
\usepackage{inputenc}
\usepackage{here}
\usepackage{graphicx}
\usepackage{paralist}
\usepackage{selinput}\usepackage[T1]{fontenc}
\usepackage[scaled]{beramono}
\usepackage{listings}
\lstset{
  language=Matlab,
  basicstyle=\ttfamily,
  showstringspaces=false,
  formfeed=newpage,
  tabsize=4,
  commentstyle=\itshape,
}
\SelectInputMappings{%
adieresis={ä},
germandbls={ß},
}
\title{\textbf{Numerik-Programmierabgabe: Blatt 6}\\
Gruppe: 3}
\author{Martin Bieker\\
		Julian Surmann\\
}
\date{}
\usepackage{graphicx}
\begin{document}
\renewcommand\tablename{Tabelle}
\renewcommand\figurename{Abbildung}
\maketitle
\section{Funktion \texttt{LU = LU\_decompose(A)}}
Diese Funktion berechnet aus einer gegebenen Matrix A die LR-Zerlegung und gibt diese platzsparend in einer Matrix LU wieder zur\"uck. Zun\"achst werden die Quotienten $q_i$ der Matrixelemente $A_{i-1,i} ... A_{n,i}$ mit dem Pivotelement $A_{i,i}$ berechnet. Danach werden die Elemente der oberen Dreiecksmatrix bestimmt.
\lstinputlisting[caption=Skript - \textit{LU\_decompose.m}]{LU_decompose.m}

\section{Funktion \texttt{z = forward\_solve(LU, b)}}
Diese Funktion löst das LGS 
\[
L\cdot \vec z = \vec b 
\]
durch Vorw\"artseinsetzen. Dabei wird ausgenutzt, dass die Elemente auf der Hauptdiagonalen der unteren Dreiecksmatrix gleich eins sind. Daher gilt f\"ur $z_i$:
\[
z_i = x_i - \sum^{i-1}_{j=1} L_{i,j}\cdot b_j \mathrm{.}
\]

\lstinputlisting[caption=Skript - \textit{forward\_solve.m}]{forward_solve.m}

\section{Funktion \texttt{x = backward\_solve(LU, z)}}

Als letzter Schritt zur Bestimmung des L\"osungsvektors $\vec x$ ist das Gleichungssystem mit der Oberen Dreiecksmatrix
\[
R\cdot\vec x = \vec z
\]
 zu l\"osen. Dazu werden die Schritte aus Abschnitt 2 r\"uckw\"arts durchlaufen. Es gilt f\"ur das $i$-te Element von $\vec x$:
 \[
 x_i = \frac{z_i- \sum^n_{j=i +1} A_{i,j}\cdot x_j }{A_{i,i}}
 \]
 \newpage
 \lstinputlisting[caption=Skript - \textit{backward\_solve.m}]{backward_solve.m}
 
 \section{Skript \texttt{LU\_test.m}}
 Um die oben implementierten Funktionen zu testen, werden in diesem Skript 3 zuf\"allige Matrizen und rechte Seiten verschiedener Gr\"o\ss e erzeugt. Danach wird die LR-Zerlegung der Matrix bestimmt. Die L\"osung des LGS wird dann mit Hilfe der Funktionen \texttt{forward\_solve} und \texttt{backward\_solve} bestimmt. Zum Vergleich werden die Gleichungssysteme auch mit dem in Octave integrierten Operator bestimmt.
 \newpage
 \lstinputlisting[caption=Skript - \textit{LU\_Test.m}]{LU_test.m}
\end{document}